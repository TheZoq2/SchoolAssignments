
\documentclass[10pt,twocolumn]{article}

% Följande rad ska göra det möjligt att använda svenska bokstäver, som å, ä, ö. Kravet är 
% då att filen sparas i UTF-8-format. Om detta inte fungerar för dig, så kan du alltid 
% använda dig av {\aa} för å, \"a för ä och \"o för ö.
\usepackage[utf8]{inputenc}

% Följande väljer typsnitt som är kloner av Times New Roman, Helvetica och lämpliga till
% dem anpassade matematiktypsnitt.
\usepackage{newtxtext}
\usepackage{newtxmath}

%  Följande tillhandahåller miljön spver­ba­tim som är lämplig för att typsätta programkod.
\usepackage{spverbatim}

\raggedbottom
\sloppy

\title{Laborationsrapport i TSKS10 \emph{Signaler, Information och Kommunikation}}

\author{Frans Skarman \\ frask812, 9509085552 }

\date{18 maj, 2017}

\begin{document}

\maketitle

\section{Inledning}

Denna laboration gick ut på att demodulera en iq modulerad signal som var given
tillsammans med andra signaler i en .wav fil. I och Q delen av signalen skulle
bestå av två separata melodier följt av två ordspråk.

Utöver den sökta signalen innehöll även datan två signaler med andra bärfrekvenser
och ett simulerat eko med okänd tidsfördröjning. Hela signalen var även tidsförskjuten
för att simulera osynkroniserade klockor på sändare och mottagare. Datan innehöll
även två stycken cosinus-signaler med närliggande frekvenser som kunde utnyttjas för
att bestämma ekots fördröjning.

I uppgiften ingick en lista på vilka bärfrekvenser huvudsignalerna kunde användas och
en maxgräns på ekots tidsfördröjning som var $500ms$. Det var även givet att amplituden
på den ekande signalen var $0.9*x(t-\delta)$ där x den skickade signalen och delta
är ekots tidsfördröjning.

Det var givet i uppgiften att den givna datans sample-rate $f_s$ var $400kHz$



\section{Metod}

% Inläsning av signaler

Alla beräkingar gjordes med python-biblioteken scipy och numpy. För att lyssna på de demodulerade
signalerna användes initialt programmet audacity men det byttes senare ut mot python-biblioteket
sounddevice.

Det första som gjordes var att fouriertransformera den givna signalen och plotta
dess spektrum. Genom att analysera spektrumet visades det att signalen bestog av tre
delsignaler med frekvenser $FREKVENSER$ samt två cosinussignaler med $FREKVENSER$.

En funktion för att göra de första stegen i demodulationen IQ signalerna skrevs. Denna
funktion multiplicerade den givna signalen med en en signal $cos(2\pi*f_c*t)$ där $f_c$
är bärfrekvensen för signalen. Den resulterande signalen filtreras sedan med ett 
butterworth-lågpassfilter med gränsfrekvens $5 kHz$. Gränsfrekvensen valdes eftersom
att signalernas bandbredd var ungefär $5 kHz$

Denna funktion användes för att demodulera de tre nyttosignalerna och resultatet spelades
upp med audacity. En av signalerna visade sig vara brus, en var fågelkvitter och en började
med vad som lät som två överlagrade melodier. Den sistnämnda signalen hade bärfrekvensen 151kHz
och är signalen som resten av rapporten kommer att analysera.

Nästa steg var att ta bort ekot från signalen vilket började med att hitta ekots tidsfördröjnign $\delta$.
För att göra detta korrelerades de $100000$ första samplen
av signalen med de efterföljande $400000$ vilket med datans sample-rate motsvarar en sekund.
$400000$ valdes eftersom att det motsvarar en sekund med den givna sample-raten,
och att det i uppgiften var givet att ekots fördröjning var mindre än $500ms$

Korrelationen gav ett maximalt värde vid sampel $63999$ vilket med den initiala förskjutningen på
$100000$ sampel gav ekots tidsfördröjning $\delta=163999$ sampel eller $0.41s$.

Eftersom att ekot inte påverkar de första $\delta$ sampeln kunde värdet på signalen $x(t+\delta)$ beräknas
som $x(t+\delta) - 0.9*x(t)$.

\section{Resultat}

Den sökta informationen är:
\begin{itemize}
\item Bärfrekvensen för nyttosignalen är $f_c=...$
\item ...
\end{itemize}

\clearpage

\section*{Min Python-kod:}
\begin{spverbatim}
clear all
close all

for k=1:...
  ...
end

plot(...,...)
\end{spverbatim}

\end{document}
